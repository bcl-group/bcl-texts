\documentclass{jsarticle}
\usepackage{amsmath}
\usepackage{geometry}
\geometry{body={165mm,230mm}}

\def\version{3.1}
\begin{document}

\begin{center}
  {\LARGE 楽しい運動計測実習その1: 筋電計測編(ver.\version)}
\end{center}
\begin{flushright}
\today
\end{flushright}

\section*{実習での注意}
\begin{itemize}
\item 取得データはデータ格納専用の外部HDDに専用フォルダを作って格納す
  ること。ローカルディスクに放置すると謎のゴミになってしまい, 後で困り
  ます。
\item 計測実験の時には、「データをとっては解析」と繰り返すのが王道。
  計測データを全てそろえてからその解析に入ると、あとで計測時の不備に気
  づいて、全データの取り直しになってしまうことがよくある。
\item 取得したどの実験データについても,どのようなデータ処理をしてどのようなグラフを作れば,説得力のあるものができるかをよく考えること。
\item 以下の実験は, 必ず全員のデータを取って解析してください。個人差が
  ないかを確認することは重要です。
\item 解析プログラムは各人作成後, お互いにその出力が同じになっているか
  を確認すること。バグ取りは重要です。
% \item 運動計測実験では,身長・体重等の身体パラメータがわかったほうが考
%   察に役立つこともありますが,体重等を秘密にしたい方は無理に測らなくて
%   もOKです。
\end{itemize}

% \section{1日目:歩行から走行}
% \subsection{計測実験}
% 以下を計測実験によって調べましょう。
% \begin{enumerate}
% \item 歩行から走行になる遷移速度はヒトによって違うだろうか?
% \item 速度が上昇するときと減速するときでは遷移速度は違うだろうか?
% \item 各移動速度で、一定時間に何回足を動かしているか数えて、
%   移動速度と足の運動周期の関係を求めてみよう。
% \item 歩容の遷移速度や足の運動周期と足の長さ等の身体パラメータの間にな
%   にか関係があるだろうか?
% \end{enumerate}

% \noindent
% [注意]
% \begin{enumerate}
% \item 実験の際は,被験者には移動速度は見せないようにすること。
% \item 各移動速度での移動になれるための若干の時間を確保すること。
% \end{enumerate}

% \subsection{レポート作成}
% \begin{enumerate}
% \item \LaTeX を使って、実験結果をレポートにしなさい。\LaTeX の使いかた
%   は裏ページと書籍を参照のこと。
% \item 何をやって、どんな結果になったかと,結果に基づく簡単な考察を書くこと。
% \item 結果をわかりやすく示す図表を作ること。グラフはxmgrace(マニュアル
%   有)かgnuplotを、表は\LaTeX の表組機能を使うこと。
% \item 注意:図の引用は\verb|\label|,\verb|\ref|を使うこと。
% \item 〆切は2日目の夕方。
% \item レポートが早くできて死にそうなヒトは、宿題をしましょう。
% \end{enumerate}

\section{高速度カメラと筋電計による運動計測その1}

\subsection{運動計測とレポート1}
上腕は鉛直下向き, 前腕を前に水平に出した状態で, いろいろな重さ(4種類以
上)のものを持ち, その重さと上腕二頭筋の筋電位の大きさの関係を**解析してレポートにまとめなさい。
なお,\textbf{どのような計測実験にすべきか,よく考えてから実験を行うこと。}

\subsection{運動計測2}

腕をまっすぐ前方に伸ばし,上腕は動かないように逆の手で支えなさい。
この状態で,前腕を水平面内で左右に動かす場合,ゆっくり動かすときと早く動かすときについて, 手先の軌道と筋電(EMG)波形を計測しなさい。

\begin{itemize}
\item 手首, 肘関節, 肩関節の位置にマーカを, 動作時に活動する筋肉(上腕
  二頭筋と上腕三頭筋)に筋電計用電極をとりつけ、運動計測をしなさい。
\item マーカをとりつける関節位置は別途資料参照。
\item 高速度カメラのサンプリングは200 frame$/$sで行うこと。
\item 各関節位置の軌跡はMoveTRで抽出する。
%\item 得られたデータは自分のホームに転送して解析すること。
\end{itemize}


\subsection{機器のセッティング}
\begin{itemize}
\item 計測用パソコン・高速度カメラ・ユニット・同期パルス発生装置・タッパ・筋電計のデータ送受信機を配置
\end{itemize}

\subsection{運動計測2のレポート}

以下をレポートにまとめなさい。ただし,どのようなグラフ(何と何の関係)を,どのように(縦軸や横軸のレンジ等)を作って説明すれば説得力のあるレポートになるかをよく考えること。


\begin{enumerate}
	\item ゆっくり腕を動かした場合について,腕の運動のどのタイミングでどの筋肉が活動するのかを調べなさい。また,その\textbf{物理学的理由}を考察しなさい。
	\item はやく腕を動かした場合についても同様の解析を行い,速度により筋活動タイミングが異なるか否かを調べなさい。また,異なるならば何故か,その\textbf{物理学的理由}を考察しなさい。
\end{enumerate}

\end{document}
