\documentclass{jarticle}
\usepackage{amsmath}
\usepackage{geometry}
\geometry{body={165mm,230mm}}

\def\version{2.1}
\begin{document}

\begin{center}
  {\LARGE 楽しい運動計測実習(プレプログラム反応版)その1(ver.\version)}
\end{center}
\begin{flushright}
\today
\end{flushright}

\section*{実習での注意}
\begin{itemize}
\item 取得データはデータ格納専用の外部HDDに専用フォルダを作って格納す
  ること。ローカルディスクに放置すると謎のゴミになってしまい, 後で困り
  ます。
\item 計測実験の時には、「データをとっては解析」と繰り返すのが王道。
  計測データを全てそろえてからその解析に入ると、あとで計測時の不備に気
  づいて、全データの取り直しになってしまうことがよくある。
\item 以下の実験は, 必ず全員のデータを取って解析してください。個人差が
  ないかを確認することは重要です。
\item 解析プログラムは各人作成後, お互いにその出力が同じになっているか
  を確認すること。バグ取りは重要です。
% \item 運動計測実験では,身長・体重等の身体パラメータがわかったほうが考
%   察に役立つこともありますが,体重等を秘密にしたい方は無理に測らなくて
%   もOKです。
\end{itemize}

\section{運動計測}
テキスト「運動神経生理学講義」(ラタッシュ著, 大修館書店)のp.264-265の「3.プレ・プログラム反応」の2つの実験を行う。

\subsection{計測実験について}
\begin{itemize}
\item テキストでは関節軌道の取得にゴニオメーターを用いているが,代わりにモーションキャプチャを用いる。
\item モーションキャプチャのサンプリングは200 frame$/$sで行うこと。
\item 各関節位置の軌跡はMoveTRで抽出する。
\item 得られたデータは自分のホームに転送して解析すること。
\end{itemize}

\subsection{データ解析}
\subsubsection{筋電データの処理の流れ}
筋電データは以下の手順で解析する。
\begin{enumerate}
\item バンドパスフィルターで雑音除去
\item 整流
\item 平滑化
\end{enumerate}

\subsubsection{筋電データの処理のプログラム作成}
\begin{enumerate}
\item \textbf{雑音除去: } octaveを使って,筋電位に60〜200 Hzの通過帯域を持つバンドパスフィルター(5次バターワース)をかける。
  (CVSを使って「octave-filter」というレポジトリを入手し,
  \verb|butter.m|, \verb|filtfilt.m|, \verb|bilinear.m|, \verb|sftrans.m|, 
  \verb|validate_filter_bands.m|を使う)
\item \textbf{整流: } C言語で筋電位$E(t)$を整流する(絶対値をとる)。
\item \textbf{平滑化: } C言語で適当な幅$\Delta T$の窓を設定して, その中での積分値を求め, 筋肉の
  活動度を見る指標とする。
\end{enumerate}
つまり, 時刻$t$における筋肉の活動度($a(t)$)は以下で評価することになる。
\begin{align}
  a(t)=\frac{1}{\Delta T}\int_{t-\Delta T/2}^{t+\Delta T/2}|E(t)|dt\notag
\end{align}
$T$は,解析する運動の速さや,注目したい運動時間に応じて変更する。

\subsubsection{運動軌道データの切取り}

\begin{enumerate}
\item 計測によって得られた身体軌道のデータの内容を確認し, フォーマットとデータ
  (数値)の意味を把握しなさい。
  フォーマットは,使うソフトウェアの種類によって異なるが,大抵以下のよ
  うになっている。
  \begin{enumerate}
  \item ヘッダ部(始め):計測条件, データの各列のタイトル
  \item データ部: シーン番号、マーカID、x座標, y座標, z座標等
  \item テール部(最後): 計測データを統計処理したデータが格納されている。
  \end{enumerate}
\item データファイルの名前を\verb|joints.csv|とするとき、このファイル
  からデータ部のみを取り出した\verb|pos-joints.dat|と、
  それ以外(ヘッダ部とテール部)のみを取り出したファイル\verb|info-joints.dat|
  を作成するシェルスクリプト\verb|getdat.sh|を作りなさい。
  各出力ファイルの仕様は以下の通りとする。
  \begin{enumerate}
  \item \verb|pos-joints.dat|は、データ部をそのままとり出す。
% \begin{verbatim}
% シーン,計測点1x,計測点2x,...,計測点1y,計測点2y,...
% \end{verbatim}
  \item \verb|info-joints.dat|は、データ部以外を取り出す。
    (もしあれば)ダブルクォーテーション(")は取り除く。
%    各行はじめの2フィールドのみを取り出す。(sedおよびcutコマンドを利用する)
  \end{enumerate}
  取得データファイルにおいて、データ部の行頭が
  数字であることを利用すれば,egrepコマンドを使ってデータ部とそれ以外をそ
  れぞれ抽出できる。
\end{enumerate}


\subsubsection{運動軌道データの処理}

データファイル\verb|pos-joints.dat|から、各関節
の各座標データを抽出する処理を自動化したい。
\begin{enumerate}
\item 以下のようなプログラム\verb|extract.c|を作りなさい。
  \begin{enumerate}
  \item 実行時には以下のような引数を指定できるようにする。
\begin{verbatim}
    $ extract <サンプリング周波数> <マーカ数> <入力データファイル名>
\end{verbatim}
  \item 出力データに書き込む「時刻」は、引数で与えた「サンプリング周波数」
    と、入力データの「シーン」番号により計算する。
  \item 入力ファイルが\verb|name.dat|のとき、出力ファイル名は計測点
    IDを使って\verb|1-name.dat|, \verb|2-name.dat|,...とする。
    もし、入力ファイル名が\verb|pos-joints.dat|ならば、出力ファイル名は計測点
    IDを使って\verb|1-pos-joints.dat|,
    \verb|2-pos-joints.dat|,...である。
  \item 各出力データファイルのフォーマットは以下の通り。
\begin{verbatim}
時刻, x座標, y座標, z座標
\end{verbatim}
  \end{enumerate}
\item  以下のようなシェルスクリプト\verb|extract.sh|を作りなさい。
  その仕様は以下の通りとする。
  \begin{enumerate}
  \item 実行時には以下のような引数を指定できるようにする。
\begin{verbatim}
$ extract.sh <入力データファイル名>
\end{verbatim}
  \item 前問の\verb|extract|コマンドを、サンプリング周波数、計測
    点数、入力データファイル名を指定して呼び出して、各関節のxyzデータを
    抽出する。
  \item サンプリング周波数とマーカ数は先につくってある
    \verb|info-*.dat|から抽出する。
    (例えばgrepとcutを用いて抽出できる。
    シェルスクリプトで,あるコマンド\verb|cmd|の実行結果を変数
    \verb|CMD|に格納するには\verb|CMD=`cmd`とすればOK|)

  \end{enumerate}
\item 以下のようなシェルスクリプト\verb|cut23|をつくりなさい。
\begin{enumerate}
\item 以下のように実行したら,入力ファイルの第2,3,4フィールド(xyz座標デー
  タ)のみを抽出したファイルを出力する。(cutコマンドを使う)
\begin{verbatim}
$ cut23 1-pos-joints.dat
\end{verbatim}
\item 上記のように入力ファイル名が\verb|1-pos-joints.dat|ならば、出力
  ファイル名は\verb|1-xy-joints.dat|とする。
\end{enumerate}
\end{enumerate}

\subsection{レポート作成}

テキストを参考にして,実験結果をレポートにまとめる。

\begin{enumerate}
\item \LaTeX を使って、実験結果をレポートにしなさい。
\item \LaTeX の使い方は裏ページ(「UNIX/LINUXの基本操作」のauctexの項目)や
  各種文献参照のこと。
\item  生データや処理したデータは xmgrace (マニュアル有)を使ってグラフ
  (x-t,y-t,x-y等)にして考察しなさい。
\item 実験の目的、手法、結果、考察をきちんと書くこと。ただし,考察は数
  行程度でOK。
\item 結果をわかりやすく示す図表を作ること。
  絵を描くにはtgifやLibreOfficeを使う。
  表は\LaTeX の表組機能を使うこと。
\item 掲載した図表は必ず本文中で引用して説明すること。
\item 図表の引用は\verb|\label|,\verb|\ref|を使った
  \LaTeX の自動引用機能を使う事。
\end{enumerate}
% \newpage
% \section{ポジションセンサによる運動計測}

% ビデオによる運動計測はビデオレートが30フレーム/sと遅いため時間分解能が
% 低いが、測定ポイント(マーカ)を多数用いることができ、マー
% カ位置以外の画像情報も同時記録できることがメリットである。
% 一方、浜松ホトニクスのポジションセンサ(PSD)を用いると、マーカ位置しか
% 計測できず、マーカは同時に4個までしか計測できないが、最高300フレーム/s
% と高速計測が可能になる。

% \subsection{運動計測実験}
% ポジションセンサを用いて以下のように計測実験を行いなさい。

\end{document}
