\documentclass{jsarticle}
\usepackage{amsmath}
\usepackage{geometry}
\geometry{body={165mm,230mm}}

\def\version{3.0py}
\begin{document}

\begin{center}
  {\LARGE 楽しい運動計測実習その1: データ解析編(Python ver.\version)}
\end{center}
\begin{flushright}
\today
\end{flushright}

\section{データ解析}
データ解析用プロクラムを以下のように作りなさい。
プログラミング言語は,特に指定がない部分はpythonかRとする。

\subsection{準備}
pythonをよく知らない人は以下でまず勉強をすること。
\begin{enumerate}
	\item pythonをLinux環境にインストール(裏ページ参照)
	\item Aidemy(https://aidemy.net/)の以下のコースのうち,よく知らないところをお勉強
	\begin{enumerate}
		\item Python入門
		\item Numpyを用いた数値計算
 		\item Pandasを用いた数値計算
 		\item Matplotlibによるデータ可視化
 		\item データクレンジング(OpenCVの部分は不要)
	\end{enumerate}
\end{enumerate}


\subsubsection{筋電データの処理}
以下のプログラムを作りなさい。
\begin{enumerate}
\item 筋電位に1〜40 Hzの通過帯域を持つバンドパスフィルター(3次バターワース)をかける。ただし,単純にバンドパスフィルターをかけるとデータのピーク位置が実際の時刻より少し遅くなることが多い。そこで,順方向と逆方向の両方から一回づつフィルター処理することで時間的なズレを補正する。この処理は,pythonならばscipyのfltflt関数を使えば自動で行われる。
\item C言語で筋電位$E(t)$を整流する(絶対値をとる)。
\item C言語で適当な幅$\Delta T$の窓を設定して, その中での平均値を求め, 筋肉の活動度を見る指標とする。
\end{enumerate}
つまり, 時刻$t$における筋肉の活動度($a(t)$)は以下で評価することになる。
\begin{align}
  a(t)=\frac{1}{\Delta T}\int_{t-\Delta T/2}^{t+\Delta T/2}|E(t)|dt\notag
\end{align}

これらの処理を行う理由を生データから考察せよ。

\subsubsection{運動軌道データの切取り}

\begin{enumerate}
\item 計測によって得られた身体軌道のデータの内容を確認し, フォーマットとデータ(数値)の意味を把握しなさい。
  フォーマットは,使うソフトウェアの種類によって異なるが,以下のようになっていることが多い
  \begin{enumerate}
  \item ヘッダ部: 計測条件, データの各列のタイトル
  \item データ部: シーン番号、マーカID、x座標, y座標, z座標等
  \item テール部(最後pa): 計測データを統計処理したデータ等が格納されていることがある。
  \end{enumerate}
\item データファイルの名前を\verb|joints.csv|とするとき、このファイル
  からデータ部のみを取り出した\verb|pos-joints.dat|と、
  それ以外(ヘッダ部とテール部)のみを取り出したファイル\verb|info-joints.dat|
  を作成するシェルスクリプト\verb|getdat.sh|を作りなさい。
  各出力ファイルの仕様は以下の通りとする。
  \begin{enumerate}
  \item \verb|pos-joints.dat|は、データ部をそのままとり出す。
% \begin{verbatim}
% シーン,計測点1x,計測点2x,...,計測点1y,計測点2y,...
% \end{verbatim}
  \item \verb|info-joints.dat|は、データ部以外を取り出す。
    (もしあれば)ダブルクォーテーション(")は取り除く。
%    各行はじめの2フィールドのみを取り出す。(sedおよびcutコマンドを利用する)
  \end{enumerate}
  取得データファイルにおいて、データ部の行頭が数字であることを利用すれば,
  egrepコマンドを使ってデータ部とそれ以外をそれぞれ抽出できる。
\end{enumerate}


\subsubsection{運動軌道データの処理}

データファイル\verb|pos-joints.dat|から、各関節の各座標データを抽出する処理を自動化したい。
\begin{enumerate}
\item 以下のようなプログラム\verb|extract.c|を作りなさい。
  \begin{enumerate}
  \item 実行時には以下のような引数を指定できるようにする。
\begin{verbatim}
    $ extract <サンプリング周波数> <マーカ数> <入力データファイル名>
\end{verbatim}
  \item 出力データに書き込む「時刻」は、引数で与えた「サンプリング周波数」
    と、入力データの「シーン」番号により計算する。
  \item 入力ファイルが\verb|name.dat|のとき、出力ファイル名は計測点
    IDを使って\verb|1-name.dat|, \verb|2-name.dat|,...とする。
    もし、入力ファイル名が\verb|pos-joints.dat|ならば、出力ファイル名は計測点
    IDを使って\verb|1-pos-joints.dat|,
    \verb|2-pos-joints.dat|,...である。
  \item 各出力データファイルのフォーマットは以下の通り。
\begin{verbatim}
時刻, x座標, y座標, z座標
\end{verbatim}
  \end{enumerate}
\item  以下のようなpythonプログラム\verb|extract.py|を作りなさい。
  その仕様は以下の通りとする。
  \begin{enumerate}
  \item 実行時には以下のような引数を指定できるようにする。
\begin{verbatim}
$ python extract.py <入力データファイル名>
\end{verbatim}
  \item 前問の\verb|extract|コマンドを、サンプリング周波数、計測点数、入力データファイル名を指定して呼び出して、各関節のxyzデータを抽出する。
  \item サンプリング周波数とマーカ数は先につくってある
    \verb|info-*.dat|から抽出する。
    (例えばgrepとcutを用いて抽出できる。
    シェルスクリプトで,あるコマンド\verb|cmd|の実行結果を変数
    \verb|CMD|に格納するには\verb|CMD=`cmd`とすればOK|)
  \end{enumerate}
\item 以下のようなシェルスクリプト\verb|cut23|をつくりなさい。
\begin{enumerate}
\item 以下のように実行したら,入力ファイルの第2,3,4フィールド(xyz座標データ)のみを抽出したファイルを出力する。(cutコマンドを使う)
\begin{verbatim}
$ cut23 1-pos-joints.dat
\end{verbatim}
\item 上記のように入力ファイル名が\verb|1-pos-joints.dat|ならば、出力
  ファイル名は\verb|1-xy-joints.dat|とする。
\end{enumerate}
\end{enumerate}

\subsection{レポート作成}
\begin{enumerate}
\item \LaTeX を使って、実験結果をレポートにしなさい。
\item \LaTeX の使い方は裏ページや各種文献参照のこと。
\item  生データや処理したデータはグラフ(x-t,y-t,x-y等)にして考察しなさい。グラフは,pythonならばmatplotlibなどを使う(他のライブラリでも良い)。
\item 実験の目的、手法、結果、考察をきちんと書くこと。ただし,考察は数行程度でOK。
\item 結果をわかりやすく示す図表を作ること。
  絵を描くにはLibreOffice等を使う。
  表は\LaTeX の表組機能を使うこと。
\item 掲載した図表は必ず本文中で引用して説明すること。
\item 図表の引用は\verb|\label|,\verb|\ref|を使った
  \LaTeX の自動引用機能を使う事。
\end{enumerate}

\end{document}
